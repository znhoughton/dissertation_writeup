% Options for packages loaded elsewhere
\PassOptionsToPackage{unicode}{hyperref}
\PassOptionsToPackage{hyphens}{url}
\PassOptionsToPackage{space}{xeCJK}
%
\documentclass[
  letterpaper,
  DIV=11,
  numbers=noendperiod,
  nottoc,
  oneside]{scrreprt}

\usepackage{amsmath,amssymb}
\usepackage{iftex}
\ifPDFTeX
  \usepackage[T1]{fontenc}
  \usepackage[utf8]{inputenc}
  \usepackage{textcomp} % provide euro and other symbols
\else % if luatex or xetex
  \usepackage{unicode-math}
  \defaultfontfeatures{Scale=MatchLowercase}
  \defaultfontfeatures[\rmfamily]{Ligatures=TeX,Scale=1}
\fi
\usepackage{lmodern}
\ifPDFTeX\else  
    % xetex/luatex font selection
    \setmainfont[]{Crimson}
  \ifXeTeX
    \usepackage{xeCJK}
    \setCJKmainfont[]{Noto Serif KR}
          \fi
  \ifLuaTeX
    \usepackage[]{luatexja-fontspec}
    \setmainjfont[]{Noto Serif KR}
  \fi
\fi
% Use upquote if available, for straight quotes in verbatim environments
\IfFileExists{upquote.sty}{\usepackage{upquote}}{}
\IfFileExists{microtype.sty}{% use microtype if available
  \usepackage[]{microtype}
  \UseMicrotypeSet[protrusion]{basicmath} % disable protrusion for tt fonts
}{}
\makeatletter
\@ifundefined{KOMAClassName}{% if non-KOMA class
  \IfFileExists{parskip.sty}{%
    \usepackage{parskip}
  }{% else
    \setlength{\parindent}{0pt}
    \setlength{\parskip}{6pt plus 2pt minus 1pt}}
}{% if KOMA class
  \KOMAoptions{parskip=half}}
\makeatother
\usepackage{xcolor}
\usepackage[left=2in,right=2in,marginparwidth=1.5in,twoside=true]{geometry}
\setlength{\emergencystretch}{3em} % prevent overfull lines
\setcounter{secnumdepth}{5}
% Make \paragraph and \subparagraph free-standing
\makeatletter
\ifx\paragraph\undefined\else
  \let\oldparagraph\paragraph
  \renewcommand{\paragraph}{
    \@ifstar
      \xxxParagraphStar
      \xxxParagraphNoStar
  }
  \newcommand{\xxxParagraphStar}[1]{\oldparagraph*{#1}\mbox{}}
  \newcommand{\xxxParagraphNoStar}[1]{\oldparagraph{#1}\mbox{}}
\fi
\ifx\subparagraph\undefined\else
  \let\oldsubparagraph\subparagraph
  \renewcommand{\subparagraph}{
    \@ifstar
      \xxxSubParagraphStar
      \xxxSubParagraphNoStar
  }
  \newcommand{\xxxSubParagraphStar}[1]{\oldsubparagraph*{#1}\mbox{}}
  \newcommand{\xxxSubParagraphNoStar}[1]{\oldsubparagraph{#1}\mbox{}}
\fi
\makeatother


\providecommand{\tightlist}{%
  \setlength{\itemsep}{0pt}\setlength{\parskip}{0pt}}\usepackage{longtable,booktabs,array}
\usepackage{calc} % for calculating minipage widths
% Correct order of tables after \paragraph or \subparagraph
\usepackage{etoolbox}
\makeatletter
\patchcmd\longtable{\par}{\if@noskipsec\mbox{}\fi\par}{}{}
\makeatother
% Allow footnotes in longtable head/foot
\IfFileExists{footnotehyper.sty}{\usepackage{footnotehyper}}{\usepackage{footnote}}
\makesavenoteenv{longtable}
\usepackage{graphicx}
\makeatletter
\def\maxwidth{\ifdim\Gin@nat@width>\linewidth\linewidth\else\Gin@nat@width\fi}
\def\maxheight{\ifdim\Gin@nat@height>\textheight\textheight\else\Gin@nat@height\fi}
\makeatother
% Scale images if necessary, so that they will not overflow the page
% margins by default, and it is still possible to overwrite the defaults
% using explicit options in \includegraphics[width, height, ...]{}
\setkeys{Gin}{width=\maxwidth,height=\maxheight,keepaspectratio}
% Set default figure placement to htbp
\makeatletter
\def\fps@figure{htbp}
\makeatother
% definitions for citeproc citations
\NewDocumentCommand\citeproctext{}{}
\NewDocumentCommand\citeproc{mm}{%
  \begingroup\def\citeproctext{#2}\cite{#1}\endgroup}
\makeatletter
 % allow citations to break across lines
 \let\@cite@ofmt\@firstofone
 % avoid brackets around text for \cite:
 \def\@biblabel#1{}
 \def\@cite#1#2{{#1\if@tempswa , #2\fi}}
\makeatother
\newlength{\cslhangindent}
\setlength{\cslhangindent}{1.5em}
\newlength{\csllabelwidth}
\setlength{\csllabelwidth}{3em}
\newenvironment{CSLReferences}[2] % #1 hanging-indent, #2 entry-spacing
 {\begin{list}{}{%
  \setlength{\itemindent}{0pt}
  \setlength{\leftmargin}{0pt}
  \setlength{\parsep}{0pt}
  % turn on hanging indent if param 1 is 1
  \ifodd #1
   \setlength{\leftmargin}{\cslhangindent}
   \setlength{\itemindent}{-1\cslhangindent}
  \fi
  % set entry spacing
  \setlength{\itemsep}{#2\baselineskip}}}
 {\end{list}}
\usepackage{calc}
\newcommand{\CSLBlock}[1]{\hfill\break\parbox[t]{\linewidth}{\strut\ignorespaces#1\strut}}
\newcommand{\CSLLeftMargin}[1]{\parbox[t]{\csllabelwidth}{\strut#1\strut}}
\newcommand{\CSLRightInline}[1]{\parbox[t]{\linewidth - \csllabelwidth}{\strut#1\strut}}
\newcommand{\CSLIndent}[1]{\hspace{\cslhangindent}#1}

\usepackage{tocbibind}  % Include TOC, List of Figures, and List of Tables in the TOC
\usepackage{fontspec}
\KOMAoption{captions}{tableheading}
\makeatletter
\@ifpackageloaded{bookmark}{}{\usepackage{bookmark}}
\makeatother
\makeatletter
\@ifpackageloaded{caption}{}{\usepackage{caption}}
\AtBeginDocument{%
\ifdefined\contentsname
  \renewcommand*\contentsname{Table of contents}
\else
  \newcommand\contentsname{Table of contents}
\fi
\ifdefined\listfigurename
  \renewcommand*\listfigurename{List of Figures}
\else
  \newcommand\listfigurename{List of Figures}
\fi
\ifdefined\listtablename
  \renewcommand*\listtablename{List of Tables}
\else
  \newcommand\listtablename{List of Tables}
\fi
\ifdefined\figurename
  \renewcommand*\figurename{Figure}
\else
  \newcommand\figurename{Figure}
\fi
\ifdefined\tablename
  \renewcommand*\tablename{Table}
\else
  \newcommand\tablename{Table}
\fi
}
\@ifpackageloaded{float}{}{\usepackage{float}}
\floatstyle{ruled}
\@ifundefined{c@chapter}{\newfloat{codelisting}{h}{lop}}{\newfloat{codelisting}{h}{lop}[chapter]}
\floatname{codelisting}{Listing}
\newcommand*\listoflistings{\listof{codelisting}{List of Listings}}
\makeatother
\makeatletter
\makeatother
\makeatletter
\@ifpackageloaded{caption}{}{\usepackage{caption}}
\@ifpackageloaded{subcaption}{}{\usepackage{subcaption}}
\makeatother
\makeatletter
\@ifpackageloaded{sidenotes}{}{\usepackage{sidenotes}}
\@ifpackageloaded{marginnote}{}{\usepackage{marginnote}}
\makeatother

\ifLuaTeX
  \usepackage{selnolig}  % disable illegal ligatures
\fi
\usepackage{bookmark}

\IfFileExists{xurl.sty}{\usepackage{xurl}}{} % add URL line breaks if available
\urlstyle{same} % disable monospaced font for URLs
\hypersetup{
  pdftitle={Multi-Word Representations in Minds and Models: Investigating Storage Mechanisms in Humans and Large Language Models},
  pdfauthor={Zachary Nicholas Houghton},
  hidelinks,
  pdfcreator={LaTeX via pandoc}}


\title{Multi-Word Representations in Minds and Models: Investigating
Storage Mechanisms in Humans and Large Language Models}
\author{Zachary Nicholas Houghton}
\date{2024-11-25}

\begin{document}
\maketitle

\pagenumbering{roman}

\renewcommand*\contentsname{Table of contents}
{
\setcounter{tocdepth}{2}
\tableofcontents
}
\listoffigures
\listoftables

\bookmarksetup{startatroot}

\chapter*{Abstract}\label{sec-abstract}
\addcontentsline{toc}{chapter}{Abstract}

\markboth{Abstract}{Abstract}

This is my abstract.

\bookmarksetup{startatroot}

\chapter*{Acknowledgements}\label{sec-acknowledgements}
\addcontentsline{toc}{chapter}{Acknowledgements}

\markboth{Acknowledgements}{Acknowledgements}

I started this journey in the middle of a pandemic that persisted
through much of my program. It is no exaggeration to say that my success
in this program is due in large, or perhaps completely, to the people
below. {\marginnote{\begin{footnotesize}Though I would like to take some
of the credit for myself.\end{footnotesize}}}

First and foremost, I would not be here if it weren't for my incredible
advisor, Dr.~Emily Morgan. Emily has been a never-ending source of
knowledge, a guiding light, and a constant source of reassurance. Emily
was charged with the non-trivial task of helping to translate my
incoherent stream of thoughts into a coherent set of ideas. She pushed
me hard, believed in me, and never let me fall behind. Words can express
neither the gratitude nor the debt that I owe to you, Emily.

I'd also like to thank many of the other brilliant minds here who have
been crucial to my development as a researcher. Specifically, I'd like
to thank Dr.~Fernanda Ferreira, Dr.~Kenji Sagae, Dr.~Santiago Barreda,
Dr.~Georgia Zellou, and Dr.~Masoud Jasbi. Over my years at UC Davis,
each of these professors has volunteered countless hours of their time
and wisdom to me, indulging my endless stream of questions.

Many of the ideas presented here have benefited in some form or another
from feedback from many brilliant graduate students. I would especially
like to thank Casey Felton, Harvey Qiu, Skyler Reese, Nicole Dodd, and
Penny Pan for their feedback on much of the work included here.

I'd also like to thank Casey, Felix, and Nora for being a strong support
system during my time here. Our Sunday shenanigans were a welcomed
escape from the tireless work of completing a PhD.

My journey in linguistics started at the University of Oregon, and I
want to thank all of the professors that supported the beginning of my
journey. I particularly want to acknowledge Dr.~Vsevolod Kapatsinski.
Volya has donated countless hours of his time to me even after his role
as my undergraduate thesis advisor was long over. He continues to be an
endless source of knowledge and inspiration and much of my knowledge and
interest in language learning comes from him. Perhaps more importantly,
however, he is a constant reminder that linguistics is \emph{fun}! Had
it not been for our meetings over the years that devolved into
ridiculous linguistic tangents, I would have burnt out long ago. I would
not be here without you, Volya.

I would also like to thank Kim 선생님. Her words of encouragement and
faith in me helped me believe in myself.

In addition, I want to thank Dr.~Melissa Baese-Berk, Dr.~Misaki Kato,
and Dr.~Zara Harmon. Aside from being both exceptional researchers and
inspirational people, each one of them was crucial to my development as
a researcher, as a linguist, and as a person. If I can become even half
the linguist they are, I'll be incredibly proud of myself.

Along with the technical and academic guidance, it also would have been
impossible to complete this PhD without the unending support I received
from my many close friends. It would take up too much space to name all
of them, but they surely know who they are.

I have been fortunate to have a strong support system in the form of of
my two sisters, Kayla and Lily. We've been through so much together. I
don't know where I would be, not just academically, but more generally
in life, had you two not been by my side.

This work would also have not been completed without the influence of my
parents. Specifically, I want to thank my mom for teaching me that the
ability to find the answer is far more important than knowing the
answer, and my dad, for teaching me the discipline and practical skills
to achieve my goals.

Last but certainly not least, thank you 보미 for being an endless source
of positivity and encouragement.

The number of people who have been indispensable in me getting here is
undoubtedly larger than is feasible to include here. To those that I
have inevitably left out, I apologize.

\bookmarksetup{startatroot}

\chapter{Introduction}\label{introduction}

\pagenumbering{arabic}

From a young age, humans are capable of generating sentences that
they've never encountered before
(\citeproc{ref-kapatsinskiChangingMindsChanging2018}{Kapatsinski 2018};
\citeproc{ref-berkoChildsLearningEnglish1958}{Berko 1958}). This ability
is largely enabled by our ability to store forms that we've learned and
compute new forms by applying knowledge of the grammar to these stored
forms (\citeproc{ref-stembergerAreInflectedForms2004}{Joseph P.
Stemberger and MacWhinney 2004};
\citeproc{ref-stembergerFrequencyLexicalStorage1986}{Joseph Paul
Stemberger and MacWhinney 1986};
\citeproc{ref-morganAbstractKnowledgeDirect2016}{Morgan and Levy 2016},
\citeproc{ref-morganModelingIdiosyncraticPreferences2015}{2015};
\citeproc{ref-berkoChildsLearningEnglish1958}{Berko 1958}). In theory,
these can be complementary forces: if a form is stored, it does not need
to be computed, and if a form can be computed, it does not have to be
stored. For example, if the word \emph{cats} is stored, then there is no
reason to compute it. On the other hand, if it can be computed (e.g., we
have learned the word \emph{cat}, and we have learned how to make
regular forms plural in English), then there may be no reason to store
it. This has been the story told by many of the early linguistic
theories, and understandably so.

\section{Computation and Storage}\label{computation-and-storage}

\section{What is Storage?}\label{what-is-storage}

\bookmarksetup{startatroot}

\chapter*{References}\label{references}
\addcontentsline{toc}{chapter}{References}

\markboth{References}{References}

\phantomsection\label{refs}
\begin{CSLReferences}{1}{0}
\bibitem[\citeproctext]{ref-berkoChildsLearningEnglish1958}
Berko, Jean. 1958. {``The Child's Learning of English Morphology.''}
\emph{{\emph{WORD}}} 14 (2-3): 150--77.
\url{https://doi.org/10.1080/00437956.1958.11659661}.

\bibitem[\citeproctext]{ref-kapatsinskiChangingMindsChanging2018}
Kapatsinski, Vsevolod. 2018. \emph{Changing Minds Changing Tools: From
Learning Theory to Language Acquisition to Language Change}. MIT Press.
\url{https://books.google.com/books?hl=en&lr=&id=YZxjDwAAQBAJ&oi=fnd&pg=PR5&dq=kapatsinski+changing+minds&ots=9bGhgkCaY0&sig=MHfWF9cbhbtMmx33a0FYSM6AMAs}.

\bibitem[\citeproctext]{ref-morganModelingIdiosyncraticPreferences2015}
Morgan, Emily, and Roger Levy. 2015. {``Modeling Idiosyncratic
Preferences : How Generative Knowledge and Expression Frequency Jointly
Determine Language Structure,''} 1649--54.

\bibitem[\citeproctext]{ref-morganAbstractKnowledgeDirect2016}
---------. 2016. {``Abstract Knowledge Versus Direct Experience in
Processing of Binomial Expressions.''} \emph{Cognition} 157: 384--402.
\url{https://doi.org/10.1016/j.cognition.2016.09.011}.

\bibitem[\citeproctext]{ref-stembergerFrequencyLexicalStorage1986}
Stemberger, Joseph Paul, and Brian MacWhinney. 1986. {``Frequency and
the Lexical Storage of Regularly Inflected Forms.''} \emph{Memory \&
Cognition} 14 (1): 17--26. \url{https://doi.org/10.3758/BF03209225}.

\bibitem[\citeproctext]{ref-stembergerAreInflectedForms2004}
Stemberger, Joseph P., and Brian MacWhinney. 2004. {``Are Inflected
Forms Stored in the Lexicon.''} \emph{Morphology: Critical Concepts in
Linguistics} 6: 107122.
\url{https://books.google.com/books?hl=en&lr=&id=bGl0aKBld3cC&oi=fnd&pg=PA107&dq=stemberger+2004+inflected&ots=RdvzVaC_NS&sig=0DJV8gUVaoZv_COZqcLXOu5_evU}.

\end{CSLReferences}




\end{document}
